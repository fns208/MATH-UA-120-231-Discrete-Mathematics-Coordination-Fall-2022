\documentclass{article}
%% This is some font management depending on the TeX “engine” being used.
%% Nothing to worry about.
\usepackage{ifxetex}
\ifxetex
\usepackage{fontspec}
\else
\usepackage[T1]{fontenc}
\usepackage[utf8]{inputenc}
\usepackage{lmodern}
\fi

%% Student: These lines describe some document metadata.

\title{Polished Proof 5}

\usepackage{etoolbox}
\author{%
	Name
	\\
	MATH-UA 120 Discrete Mathematics
}
\date{Due: Friday, December 2, 11pm on Gradescope.}


%% These lines set up the question, answer, and solution environments.
\usepackage{amsthm}
\usepackage{amssymb}
\usepackage{amsmath}
\theoremstyle{definition}
\newtheorem{question}{Question}

\newenvironment{answer}[1][Answer]
{\begin{proof}[#1]\renewcommand\qedsymbol{$\vartriangle$}}
	{\end{proof}}
\newenvironment{solution}[1][Solution]
{\begin{proof}[#1]\renewcommand\qedsymbol{$\blacktriangle$}}
	{\end{proof}}
\makeatletter
\newcommand{\stepenumdepth}{\advance\@enumdepth\@ne}
\makeatother
\AtBeginEnvironment{question}{\stepenumdepth}
\AtBeginEnvironment{answer}{\stepenumdepth}
\AtBeginEnvironment{solution}{\stepenumdepth}

\usepackage{tikz}
\usetikzlibrary{calc}
\usetikzlibrary{positioning}
\usetikzlibrary{patterns}
\usetikzlibrary{matrix}
\tikzstyle{vertex}=[circle,draw,fill=none,inner sep=0pt,outer sep=0pt, minimum width=1ex]
\tikzstyle{edge}=[draw,thick]
\usepackage{array}

\usepackage{enumerate}

\usepackage{hyperref}
%% This is the beginning of the part of the file that describes
%% the actual text of the document.
%% That's why it says `\begin{document}' below. :-)
\begin{document}
	\maketitle
	
	\section*{Directions}

Complete the assignment in \LaTeX~  on Overleaf, download the pdf and upload on Gradescope.


\section*{Proof Options}

Please choose \textbf{one} of the following exercises. Begin with ``Claim:" and write the statement you intend to prove. Then write ``Proof:" and the proof. You can choose your own end-of-proof marker for flair.

\begin{enumerate}
	\item  Prove The Division Theorem by induction on $a$: Let $a, b\in\mathbb{Z}$ with $a\geq 0$ and $b>0$. There exist two unique integers $q$ and $r$ such that 
	\[ a=qb+r \quad \text{ and } \quad 0\leq r<b.\] 
	\item Prove by induction on $n$: For $n\in \mathbb{N}$ where $n\geq 1$, let $A_1, A_2, \dots, A_n$ be any collection of events. Then 
	\[ P\left( A_1\cup A_2 \cup \cdots \cup A_n\right) \leq \sum_{i=1}^n P(A_i).\]
	\item Prove by induction on $n$: If $n\geq 2$ and $f_1, f_2, \dots, f_n$ are invertible functions on some nonempty set $A$, then 
	\[ (f_1 \circ f_2 \circ \cdots \circ f_n)^{-1} = f_n^{-1} \circ \cdots \circ f_2^{-1} \circ f_1^{-1}.\]
\end{enumerate}

\section*{Reflection Prompt}
How did the ideas of this class enlarge your sense of what it means to do mathematics?

\section*{Grading Rubric}
This assignment will be graded on a scale of 1-10 via the RVF rubric (9 points) discussed in class. More information on this rubric can be found \href{https://drive.google.com/file/d/1P0OBjw-GkX64uCpYcqYmXARapf9MwaiI/view?usp=sharing}{here}. \href{https://drive.google.com/file/d/1KAFQ7GBFpfUkyTBRZ30h5o6nXWwYDSML/view?usp=sharing}{Here} are some examples of past Polished Proof graded work to make sure expectations are clear. \textbf{The remaining point will be given for a thoughtful and concise reflection.}


	

\end{document}

