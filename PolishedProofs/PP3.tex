\documentclass{article}
%% This is some font management depending on the TeX “engine” being used.
%% Nothing to worry about.
\usepackage{ifxetex}
\ifxetex
\usepackage{fontspec}
\else
\usepackage[T1]{fontenc}
\usepackage[utf8]{inputenc}
\usepackage{lmodern}
\fi

%% Student: These lines describe some document metadata.

\title{Polished Proof 3}

\usepackage{etoolbox}
\author{%
	Name
	\\
	MATH-UA 120 Discrete Mathematics
}
\date{Due: Friday, October 21, 11pm on Gradescope.}


%% These lines set up the question, answer, and solution environments.
\usepackage{amsthm}
\usepackage{amssymb}
\usepackage{amsmath}
\theoremstyle{definition}
\newtheorem{question}{Question}

\newenvironment{answer}[1][Answer]
{\begin{proof}[#1]\renewcommand\qedsymbol{$\vartriangle$}}
	{\end{proof}}
\newenvironment{solution}[1][Solution]
{\begin{proof}[#1]\renewcommand\qedsymbol{$\blacktriangle$}}
	{\end{proof}}
\makeatletter
\newcommand{\stepenumdepth}{\advance\@enumdepth\@ne}
\makeatother
\AtBeginEnvironment{question}{\stepenumdepth}
\AtBeginEnvironment{answer}{\stepenumdepth}
\AtBeginEnvironment{solution}{\stepenumdepth}

\usepackage{tikz}
\usetikzlibrary{calc}
\usetikzlibrary{positioning}
\usetikzlibrary{patterns}
\usetikzlibrary{matrix}
\tikzstyle{vertex}=[circle,draw,fill=none,inner sep=0pt,outer sep=0pt, minimum width=1ex]
\tikzstyle{edge}=[draw,thick]
\usepackage{array}

\usepackage{enumerate}

\usepackage{hyperref}
%% This is the beginning of the part of the file that describes
%% the actual text of the document.
%% That's why it says `\begin{document}' below. :-)
\begin{document}
	\maketitle
	
	\section*{Directions}

Complete the assignment in \LaTeX~  on Overleaf, download the pdf and upload on Gradescope.


\section*{Proof Options}

Please choose \textbf{one} of the following exercises. Begin with ``Claim:" and write the statement you intend to prove. Then write ``Proof:" and the proof. You can choose your own end-of-proof marker for flair. \textbf{You may not manipulate the question algebraically.}

\underline{Prove the following {\bf combinatorially}.}
\begin{enumerate}
    \item Let $k, m, n$ be positive integers such that $n\geq k\geq m\geq 0$.  Then,
    \[ {n\choose k} {k\choose m} = {n\choose m} {{n-m}\choose {k-m}}. \]

    \item Let $n$ be a positive integer.  Then,
    \[ {{2n+2}\choose {n+1}} = {{2n}\choose {n+1}} +2 {{2n}\choose {n}} + {{2n}\choose {n-1}}. \]
\end{enumerate}


\section*{Grading Rubric}
This assignment will be graded on a scale of 1-10 via the RVF rubric (9 points) discussed in class. More information on this rubric can be found \href{https://drive.google.com/file/d/1P0OBjw-GkX64uCpYcqYmXARapf9MwaiI/view?usp=sharing}{here}. \href{https://drive.google.com/file/d/1KAFQ7GBFpfUkyTBRZ30h5o6nXWwYDSML/view?usp=sharing}{Here} are some examples of past Polished Proof graded work to make sure expectations are clear. The remaining point will be given for the proper use of \LaTeX.
	

\end{document}

