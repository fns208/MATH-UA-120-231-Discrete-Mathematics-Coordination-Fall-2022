\documentclass{article}
% This is a LaTeX file.  It is a text file that is compiled
% by a program called LaTeX into a pretty PDF file.  
% If you're viewing this file on CoCalc, you'll see that PDF 
% in the window to the right.
%
% The LaTeX macro language is complicated, so we have inserted
% lots of documenting comments into the file.  Comments start
% with `%' and continue to the end of the line.  In CoCalc's
% window, they are colored brownish-red.
%
% Comments prefixed with `Student:' are relevant to students.
% Skip anything else you don't understand, or ask me.
%
% set font encoding for PDFLaTeX or XeLaTeX
\usepackage{ifxetex}
\ifxetex
  \usepackage{fontspec}
\else
  \usepackage[T1]{fontenc}
  \usepackage[utf8]{inputenc}
  \usepackage{lmodern}
\fi

% Student: These lines describe some document metadata.
\title{Homework Assignment 1}
\author{%
% Student: change the next line to your name!
    Prof. Shum
\\  MATH-UA 120 Discrete Mathematics
}
\date{due February 8, 2018 at 11:00pm}


\usepackage[headings=runin-fixed-nr]{exsheets}
% These make enumerates within questions start at the second ("(a)") level, rather than the first ("1.") level.
\makeatletter
    \newcommand{\stepenumdepth}{\advance\@enumdepth\@ne}
\makeatother
\SetupExSheets{
    question/pre-body-hook=\stepenumdepth,
    solution/pre-body-hook=\stepenumdepth,
}
\DeclareInstance{exsheets-heading}{runin-nn-np}{default}{
    runin = true,
    title-post-code = .\space,
    join = {
        main[r,vc]title[l,vc](0pt,0pt);
    }
}
\newif\ifshowsolutions
% Student: replace `false' with `true' to typeset your solutions.
% Otherwise they are ignored!
\showsolutionstrue
\ifshowsolutions
    \SetupExSheets{
        question/pre-hook=\itshape,
        solution/headings=runin-nn-np,
        solution/print=true,
        solution/name=Answer
    }%
    \makeatletter%
    \pretocmd{\@title}{Answers to }%
    \makeatother%
\else
    \SetupExSheets{solution/print=false}
\fi

% Bug workaround: http://tex.stackexchange.com/a/146536/1402
%\newenvironment{exercise}{}{}
\RenewQuSolPair{question}{solution}
%\let\answer\solution
%\let\endanswer\endsolution
\usepackage{manfnt}
\newcommand{\danger}{\marginpar[\hfill\dbend]{\dbend\hfill}}

\usepackage{amsmath, amsthm}
\usepackage{amsfonts}
\usepackage{siunitx}
\DeclareSIUnit\pound{lb}
\usepackage{hyperref}
\newtheorem*{theorem}{Theorem}
\theoremstyle{definition}
\newtheorem*{definition}{Definition}
% We are creating a command "\xor".
\newcommand{\xor}{\underline{\lor}}
% This is the beginning of the part of the file that describes
% the text of the document.
% That's why it says `\begin{document}' below. :-)
\begin{document}
\maketitle



These are to be written up and turned in to Gradescope.\\



\ifshowsolutions
    \SetupExSheets{solution/print=true}
\else
    \danger
 \underline{ \LaTeX  Instructions:}  You can view the source (\texttt{.tex}) file to get some more examples of \LaTeX{} code.  I have commented the source file in places where new \LaTeX{} constructions are used.
  
  Remember to change \verb|\showsolutionsfalse| to \verb|\showsolutionstrue|
    in the document's preamble 
    (between \verb|\documentclass{article}| and \verb|\begin{document}|)
\fi

\section*{Assigned Problems}

\begin{question}
Determine if the following statement is true or false and support your reasoning. \emph{Your reasoning must be specific to this statement.}
\begin{quote}
``Every positive integer is either prime or composite.''
\end{quote}
\end{question}
% Student: put your answer between the next two lines.
\begin{solution}
Note that $1$ is a positive integer. By the definition of prime, we only consider integers greater than one; hence $1$ is not prime. By the definition of composite, we would need to find an integer $b$ such that $1<b<1$ and $b\mid 1$. Since no such $b$ exists, $1$ is not considered a composite. Since $1$ is neither prime nor composite, the statement is false.
\end{solution}

\begin{question}
    % Notice the use of the enumerate environment
    % to make a numbered list.  Each item is marked
    % by the \item command.
    %
    % Also \emph = emphasize, usually in italics.
    \begin{enumerate}
        \item 
        Suppose that the concept of points and lines
        in the plane is already defined.  Write a careful
        definition for two lines to be \emph{parallel}. 
        Your definition should begin as
        follows:
        % What do you think center does?
        \begin{center}
        Suppose $l$ and $m$ are lines in the plane. We say that $l$ and $m$ are \emph{parallel} provided...
        \end{center}

        {\emph{Note:} Since you are crafting this definition, you have a bit of flexibility.  Consider excluding the possibility that the line $l$ might be the same as the line $m$. Be sure that your definition does what you intend.}

        \item Using your definition of \emph{parallel}, explain why the following lines are \textbf{not} parallel: $y=2x+1$ and $y=x+2$.
    \end{enumerate}
\end{question}
% Student: put your answer between the next two lines.
\begin{solution}
\begin{enumerate}
\item \begin{definition}Suppose $l$ and $m$ are (distinct) lines in the plane. We say that $l$ and $m$ are \emph{parallel} provided they have no points in common.\end{definition}
\item Note that the point $(1, 3)$ is on both lines $y=2x+1$ and $y=x+2$. Since the two lines have a point in common, they are not parallel.
\end{enumerate}
\end{solution}


\begin{question}
   (Scheinerman, Exercise 4.2:)
    Below you will find pairs of statements $A$ and $B$. For each pair, please indicate which
of the following three sentences are true and which are false:
    \begin{itemize}
        \item If $A$, then $B$.
        \item If $B$, then $A$.
        \item $A$ if and only $B$.
    \end{itemize}
    \begin{enumerate}
        \item $A$: Polygon $PQRS$ is a rectangle. 
              $B$: Polygon $PQRS$ is a parallelogram.
        \item $A$: Joe is a grandfather.
              $B$: Joe is male.

         For the remaining items, $x$ and $y$ refer to \emph{real} numbers.

         \item $A$: $x>0$
               $B$: $x^2>0$
         \item $A$: $x<0$
               $B$: $x^3<0$
               
    \end{enumerate}
\end{question}
% Student: put your answer between the next two lines.
\begin{solution}
    \begin{enumerate}
        \item All rectangles are parallelograms, but not all parallelograms are rectangles.
              So ``If $A$, then $B$ is true, ``If $B$, then $A$ '' is false, and ``$A$ if and only if $B$'' is false.
        \item All grandfathers are male, but not all males are grandfathers.
              So ``If $A$, then $B$ is true, ``If $B$, then $A$ '' is false, and ``$A$ if and only if $B$'' is false.   
                       \item All positive numbers have positive squares.  But $(-3)^2 = 9 > 0$, yet $-3 < 0$.
              So ``If $A$, then $B$ is true, ``If $B$, then $A$ '' is false, and ``$A$ if and only if $B$'' is false.
         \item These are equivalent statements.
              ``If $A$, then $B$ is true, ``If $B$, then $A$ '' is true, and ``$A$ if and only if $B$'' is true.   
       \end{enumerate}
\end{solution}

\begin{question}
    (Scheinerman, Exercise 5.11:)
    Suppose $a$, $b$, $d$, $x$, and $y$ are integers.
    Prove that if $d \mid a$ and $d\mid b$,
    then $d \mid (ax+by)$.
\end{question}
% Student: put your answer between the next two lines.
\begin{solution}
Suppose that $d|a$ and $d|b$.  By the definition of divisibility, there is an integer $m$ such that $a = md$ and an integer $n$ such that $b = nd$.  Therefore, we can write $ax + by$ in terms of $d, m$, and $n$ as follows:
\begin{eqnarray*} 
ax + by & = & (md)x + (nd)y \\
	& = & d(mx + ny).
\end{eqnarray*}
Since $mx + ny$ is an integer, then by the definition of divisibility, $d | (ax + by)$.\\
\end{solution}


\begin{question}
Consider the following definition of the ``$\triangleleft$'' symbol.
	\begin{definition}
	Let $x$ and $y$ be integers. Write $x\triangleleft y$ if $3x+5y=7k$ for some integer $k$.
	\end{definition}
\begin{enumerate}
\item Show that $1\triangleleft 5$, $3\triangleleft 1$, and $0\triangleleft 7$.
\item Prove that if $a\triangleleft b$ and $c\triangleleft d$, then $(a+c) \triangleleft (b+d)$.
\end{enumerate}
\end{question}
% Student: put your answer between the next two lines.
\begin{solution}
\begin{enumerate}
\item We want to find an integer $k$ that satisfies $3x+5y=7k$.
	\begin{itemize}
	\item For $1\triangleleft 5$, we need $k=4$ such that $3(1)+5(5)=7(4)$.
	\item For $3\triangleleft 1$, we need $k=2$ such that $3(3)+5(1)=7(2)$.
	\item For $0\triangleleft 7$, we need $k=5$ such that $3(0)+5(7)=7(5)$.
	\end{itemize}
\item 
	\begin{itemize}
	\item Suppose that $a\triangleleft b$ and $c\triangleleft d$. 
	\item Then there exists integers $k_1$ and $k_2$ such that $3a+5b=7k_1$ and $3c+5d=7k_2$.
	\item \begin{eqnarray*}
	(3a+5b) + (3c+5d) &=& 7k_1 + 7k_2\\
	3(a+c) + 5(b + d) &=& 7(k_1+k_2)
	\end{eqnarray*}
	\item Let $k=k_1+k_2$. Since there exists an integer $k$ such that $$3(a+c) + 5(b + d)=7k,$$ $(a+c) \triangleleft (b+d)$.
	\end{itemize}
\end{enumerate}
\end{solution}

\begin{question}
    Show that an integer $n$ is odd if and only if $2n+2$ is divisible by 4.
\end{question}
% Student: put your answer between the next two lines.
\begin{solution}
This statement can be written as 
	
	\begin{center}
		$n \in \mathbb{Z}$ is odd $\iff 4 ~|~ (2n+2)$.
	\end{center}
	
 ($\Rightarrow$) Assume first that $n$ is odd. Then we can write $n = 2k+1$ for some integer $k$. Then
	\[
	2 n + 2 = 2 (2k + 1) + 2 = 4 k + 2 + 2 = 4 k + 4 = 4 (k+1).
	\]
	Since $k+1 \in \mathbb{Z}$, it means by definition that $2 n + 2$ is divisible by 4.
	
	($\Leftarrow$)  Assume now that $2n + 2$ is divisible by 4. Then we can write $2n + 2 = 4k$ for some $k \in \mathbb{Z}$. Simplifying gives $n + 1 = 2k$, and thus
	\[
	n = 2 k - 1 = 2 (k - 1) + 1.
	\]
	Since $k - 1 \in \mathbb{Z}$, this means by definition that $n$ is odd.
\end{solution}

\begin{question}
    \begin{enumerate}
        \item Disprove: For all real numbers $x$, $x^2\geq x$.
        \item Disprove: For every positive nonprime integers $n$, if some prime $p$ divides $n$, then some other prime $q$ ($q\neq p$) also divides $n$.
    \end{enumerate}
\end{question}
% Student: put your answer between the next two lines.
\begin{solution}
\begin{enumerate}
\item Consider $x=\frac{1}{2}$, which is a real number. Note that $\left(\frac{1}{2}\right)^2=\frac{1}{4}$ and $\frac{1}{4}< \frac{1}{2}$. Therefore the statement $x^2\geq x$ is false.

\item Consider $n=9$, which is a positive nonprime integer. Notice that the prime number $p=3$ divides 9. But since $9=3^2$, no other prime number divides 9. Therefore the statement is false.
\end{enumerate}
\end{solution}

\begin{question}
   (Scheinerman, Exercise 7.12:)
    Another method to prove that certain Boolean formulas
    are tautologies is to use the properties in Theorem~7.2
    together with the fact that $x \rightarrow y$ is equivalent
    to $(\neg x) \vee y$ (Proposition~7.3)
    For example, Exercise 7.11, part (b) asks you to establish that the formula $(x \wedge (x \rightarrow y)) \rightarrow y$ is 
    a tautology.  Here is a derivation of that fact:
    % This is an "align" environment for aligning
    % mathematical equations.  Ampersands (&) denote 
    % separation between columns.  The first one means
    % there will be alignment around the equal sign.
    % The double-ampersands put a second column, left-
    % justified.  Double-backslashes (\\) separate lines.
    \begin{align*}
        (x \wedge (x \rightarrow y)) \rightarrow y
        &= [x \wedge (\neg x \vee y)] \rightarrow y
        && \text{translate $\rightarrow$} 
        \\
        &= [(x \wedge \neg x) \vee (x \wedge y)] \rightarrow y
        && \text{distributive}
        \\
        &= [\mathrm{FALSE} \vee (x\wedge y)] \rightarrow y
        && \text{inverse elements}
        \\
        &= (x\wedge y) \rightarrow y
        && \text{identity element}
        \\
        &= \neg(x\wedge y) \vee y
        && \text{translate $\rightarrow$}
        \\
        &= (\neg x \vee \neg y) \vee y
        && \text{De~Morgan's laws}
        \\
        &= \neg x \vee (\neg y \vee y)
        && \text{associativity}
        \\
        &= \neg x \vee \mathrm{TRUE}
        && \text{inverse elements}
        \\
        &= \mathrm{TRUE}
        && \text{identity element}
        \\
    \end{align*}
    Use this technique [not truth tables] to prove that these formulas
    are tautologies:
    \begin{enumerate}
        \item $(x \rightarrow \mathrm{FALSE}) \rightarrow \neg x$
        \item $(x \rightarrow y) \wedge (x \rightarrow \neg y) \rightarrow \neg x$
    \end{enumerate}
\end{question}
% Student: put your answer between the next two lines.
\begin{solution}
\begin{enumerate}
        \item
        \begin{align*}
            (x \rightarrow \mathrm{FALSE}) \rightarrow \neg x
            &=\neg(\neg x \vee \mathrm{FALSE}) \vee \neg x
          \\&=(x \wedge \mathrm{TRUE}) \vee \neg x 
          \\&=(x \vee \neg x)\wedge(\mathrm{TRUE} \vee \neg x) 
          \\&= \mathrm{TRUE} \wedge \mathrm{TRUE} = \mathrm{TRUE}
        \end{align*}
        \item
        \begin{align*}
            (x \rightarrow y) \wedge (x \rightarrow \neg y) \rightarrow \neg x
            &= \neg((\neg x \vee y) \wedge (\neg x \vee \neg y)) \vee \neg x
          \\&= (\neg(\neg x \vee y) \vee \neg(\neg x \vee \neg y)) \vee \neg x
          \\&= ((x \wedge \neg y) \vee (x \wedge y)) \vee \neg x
          \\&= (x \wedge (\neg y \vee y))\vee \neg x
          \\&= (x \wedge \mathrm{TRUE}) \vee \neg x
          \\&= x \vee \neg x = \mathrm{TRUE}
        \end{align*}
    \end{enumerate}
\end{solution}




\begin{question}
   (Scheinerman, Exercise 7.16:)
Here is another Boolean operation called \textit{exclusive or}; it is denoted by the symbol $\xor$. It is defined in the following table.
	% This is an "array" environment for aligning
	% individual objects. It is similar to "align"
	% Right after we \begin{array}, we indicate
	% how we want to align each individual column:
	% "c" for centered, "r" for right-justified,
	% "l" ("ell") for left-justified.
	% The total number of c's, r's, and l's should match
	% the number of columns.
	% We can use this to form a table. To create the borders,
	% we use "|" (the notation for "divides") to create a
	% vertical line between each column.
	% After each line, double-backslashes (\\) separate lines.
	% If we want a horizontal line after each line, 
	% we use "\hline".
	%For math environment, we can use "$$" or "\[ \]".
	\[\begin{array}{| c | c || c |}
	\hline
	x & y & x \xor y \\
	\hline
	T & T & F \\ 
	T & F & T \\
	F & T & T \\
	F & F & F \\
	\hline
	\end{array}\]
    \begin{enumerate}
        \item Prove that $x \xor y$ is logically equivalent to $(x \wedge \neg y) \vee ((\neg x) \wedge y)$.
        \item Prove that $x \xor y$ is logically equivalent to $(x \vee y) \wedge (\neg (x \wedge y))$.
        \item Explain why the operation $\xor$ is called \textit{exclusive or}.
    \end{enumerate}
\end{question}
% Student: put your answer between the next two lines.
\begin{solution}
	\begin{enumerate}
	\item 
	\[\begin{array}{| c | c || c | c | c |}
	\hline
	x & y & x\wedge \neg y & (\neg x)\wedge y & (x \wedge \neg y) \vee ((\neg x) \wedge y)\\
	\hline
		T & T & F & F & F\\
		T & F & T & F & T\\
		F & T & F & T & T\\
		F & F & F & F & F\\
	\hline
	\end{array}\]
	\item 
	\[\begin{array}{| c | c || c | c | c |}
	\hline
	x & y & x\vee y & \neg (x \wedge y) & (x \vee y) \wedge (\neg (x \wedge y))\\
	\hline
		T & T & T & F & F\\
		T & F & T & T & T\\
		F & T & T & T & T\\
		F & F & F & T & F\\
	\hline
	\end{array}\]
	\item It is called ``\textit{exclusive or}'' because it captures the exclusive nature of the English ``\textit{or}'': you have have option A or B, \textit{but not both}. In the same way $x\xor y$ when $x$ or $y$ \textit{but not both} are $\mathtt{TRUE}$.
	\end{enumerate}
\end{solution}

\begin{question}
	Let the following statements be given.
		\begin{align*}
		p &= \text{``Jeremiah is hungry.''}\\
		q &= \text{``The refrigerator is empty.''}\\
		r &= \text{``Jeremiah is mad.''}
		\end{align*}
	\begin{enumerate}
		\item Rewrite the following statement as a Boolean expression.\\
			\begin{quote}
			If Jeremiah is hungry and the refrigerator is empty, then Jeremiah is mad.
			\end{quote}
		\item Construct a truth table for the statement in part (a).
		\item Suppose that the statement given in part (a) is true, and suppose also that Jeremiah is not mad and the refrigerator is empty. Is Jeremiah hungry? Justify your answer using the truth table.
	\end{enumerate}
\end{question}
% Student: put your answer between the next two lines.
\begin{solution}
	\begin{enumerate}
	\item $p\wedge q \to r$
	\item 
	\[\begin{array}{| c | c | c || c | c |}
	\hline
	p & q & r & p \wedge q & p \wedge q \to r\\
	\hline
		T & T & T & T & T\\
		T & T & F & T & F\\
		T & F & T & F & T\\
		T & F & F & F & T\\
		F & T & T & F & T\\
		F & T & F & F & T\\
		F & F & T & F & T\\
		F & F & F & F & T\\
	\hline
	\end{array}\]
	\item The only row where $r=False, q=True$, and $p \wedge q \to r=True$ is row 6 where $p=False$. Therefore, Jeremiah is not hungry.
	\end{enumerate}
\end{solution}

\end{document}
