\documentclass{article}
% The LaTeX macro language is complicated, so we have inserted
% lots of documenting comments into the file.  Comments start
% with `%' and continue to the end of the line.  In CoCalc's
% window, they are colored brownish-red.
%
% Comments prefixed with `Student:' are relevant to students.
% Skip anything else you don't understand, or ask me.
%
% set font encoding for PDFLaTeX or XeLaTeX
\usepackage{ifxetex}
\ifxetex
  \usepackage{fontspec}
\else
  \usepackage[T1]{fontenc}
  \usepackage[utf8]{inputenc}
  \usepackage{lmodern}
\fi

% Student: These lines describe some document metadata.
\title{Problem Set 3}
\author{%
% Student: change the next line to your name!
    Name
\\  MATH-UA 120 Discrete Mathematics
}
\date{due October 28, 2022 at 11:00pm}


\usepackage[headings=runin-fixed-nr]{exsheets}
% These make enumerates within questions start at the second ("(a)") level, rather than the first ("1.") level.
\makeatletter
    \newcommand{\stepenumdepth}{\advance\@enumdepth\@ne}
\makeatother
\SetupExSheets{
    question/pre-body-hook=\stepenumdepth,
    solution/pre-body-hook=\stepenumdepth,
}
\DeclareInstance{exsheets-heading}{runin-nn-np}{default}{
    runin = true,
    title-post-code = .\space,
    join = {
        main[r,vc]title[l,vc](0pt,0pt);
    }
}
\newif\ifshowsolutions
% Student: replace `false' with `true' to typeset your solutions.
% Otherwise they are ignored!
\showsolutionstrue
\ifshowsolutions
    \SetupExSheets{
        question/pre-hook=\itshape,
        solution/headings=runin-nn-np,
        solution/print=true,
        solution/name=Answer
    }%
    \makeatletter%
    \pretocmd{\@title}{Answers to }%
    \makeatother%
\else
    \SetupExSheets{solution/print=false}
\fi

% Bug workaround: http://tex.stackexchange.com/a/146536/1402
%\newenvironment{exercise}{}{}
\RenewQuSolPair{question}{solution}
%\let\answer\solution
%\let\endanswer\endsolution
\usepackage{manfnt}
\newcommand{\danger}{\marginpar[\hfill\dbend]{\dbend\hfill}}

% We are creating a command for some common commands.
\newcommand{\Z}{\mathbb{Z}}
\newcommand{\modulo}{\text{mod }}

% This package is for specifying graphics.  It's amazing.
% Manual at http://texdoc.net/texmf-dist/doc/generic/pgf/pgfmanual.pdf
\usepackage{tikz}

\usepackage{amsmath, amsthm}
\usepackage{amsfonts}
\usepackage{siunitx}
\DeclareSIUnit\pound{lb}
\usepackage{hyperref}
\newtheorem*{theorem}{Theorem}
\theoremstyle{definition}
\newtheorem*{definition}{Definition}
% This is the beginning of the part of the file that describes
% the text of the document.
% That's why it says `\begin{document}' below. :-)
\begin{document}
\maketitle



These are to be written up in \LaTeX{} and turned in to Gradescope.\\



\ifshowsolutions
    \SetupExSheets{solution/print=true}
\else
    \danger
 \underline{ \LaTeX  Instructions:}  You can view the source (\texttt{.tex}) file to get some more examples of \LaTeX{} code.  I have commented the source file in places where new \LaTeX{} constructions are used.
  
  Remember to change \verb|\showsolutionsfalse| to \verb|\showsolutionstrue|
    in the document's preamble 
    (between \verb|\documentclass{article}| and \verb|\begin{document}|)
\fi

\section*{Assigned Problems}

\begin{question}
    (Scheinerman, Exercise 14.15:)
    Prove: A relation $R$ on a set $A$ is antisymmetric if 
    \[ R \cap R^{-1} \subseteq \{ (a, a) ~:~ a \in A \}. \]
\end{question}
% Student: put your answer between the next two lines.
\begin{solution}
Suppose that $R \cap R^{-1} \subseteq \{ (a, a) ~:~ a \in A \}$. Consider elements $x, y\in A$ where $(x, y), (y, x) \in R$. Since $(x, y) \in R$, then $(y, x) \in R^{-1}$, and since $(y, x) \in R$, then $(x, y) \in R^{-1}$. This means that $(x, y) \in R\cap R^{-1}$ and $(y, x) \in R\cap R^{-1}$. Since $(x, y) \in R \cap R^{-1}\subseteq \{ (a, a) ~:~ a \in A \}$, then $(x, y) \in \subseteq \{ (a, a) ~:~ a \in A \}$.  This means that $x = y$. Since $(x, y), (y, x) \in R$ implies $x= y$, then $R$ is antisymmetric.
\end{solution}


\begin{question}
    (Scheinerman, Exercise 14.16:)
    \begin{enumerate}
        \item Give an example of a relation on a set that is both symmetric and transitive, but not reflexive.
        \item Explain what is wrong with the following `proof':
        
        \fbox{\parbox{\linewidth}{
    	\textbf{Statement:} If $R$ is symmetric and transitive, then $R$ is reflexive.

	\textbf{``Proof'':} Suppose $R$ is symmetric and transitive.  Symmetric means that $x \ R y$ implies $y \ R\ x$.  We apply transitivity to $x \ R \ y$ and $y \ R \ x$ to give $x \ R \ x$.  Therefore, $R$ is reflexive.
        }}
    \end{enumerate}
\end{question}
% Student: put your answer between the next two lines.
\begin{solution}
    \begin{enumerate}
    \item One possible example among numerous examples: Consider the set $A = \{1, 2, 3\}$.  The relation $R = \{ (1, 1), (1, 2), (2, 1), (2, 2) \}$ is symmetric (if $(x, y) \in R$ then $(y, x) \in R$ as well), transitive ($\forall x, y,z \in A$ where $(x, y)\in R$ and $(y, z) \in R$, then $(x, z) \in R$ also), but not reflexive since $(3, 3) \notin R$.
    \item To show that a relation is reflexive, we want $(x, x)$ to be in $R$ \underline{for all} $x \in A$.  The proof above says that if $(x, y) \in R$ (which also means that $(y, x)$ is also in $R$) , then it must be the case that $x \in R$ (and $y \in R)$.  

    However, if an element $z$ does not appear in any pair, then $(z, z) \notin R$ and hence $R$ is not reflexive.
    \end{enumerate}
\end{solution}


\begin{question}
    (Scheinerman, Exercise 15.4:)
    \begin{enumerate}
        \item Prove that if $x$ and $y$ are both odd integers, then $x \equiv y \ (\modulo 2)$.
        \item Prove that if $x$ and $y$ are both even integers, then $x \equiv y \ (\modulo 2)$.
    \end{enumerate}
\end{question}
% Student: put your answer between the next two lines.
\begin{solution}
    \begin{enumerate}
    \item Suppose that $x$ and $y$ are both odd.  Then, there exist integers $m, n$ such that $x = 2m + 1$ and $y = 2n +1$.  Therefore, $x-y = (2m + 1) - (2n+1) = 2(m-n)$.  This means that $2|x-y$.  Therefore, $x \equiv y (\text{mod } 2)$.
    \item Suppose that $x$ and $y$ are both even.  Then, there exist integers $m, n$ such that $x = 2m$ and $y = 2n$.  Therefore, $x-y = 2m - 2n = 2(m-n)$.  This means that $2|x-y$.  Therefore, $x \equiv y (\text{mod } 2)$.
    \end{enumerate}
\end{solution}


\begin{question}
    Consider the set $A = \{0, 1, 2, \dots, 8 \}$. Define a relation $R$ on $A$ by
	\[
	a R b \iff a^2 \equiv b^2 \; [9].
	\]
	\begin{enumerate}
	\item Show that $R$ is an equivalence relation, 
	\item then determine all its (distinct) equivalence classes.
	\end{enumerate}
\end{question}
% Student: put your answer between the next two lines.
\begin{solution}
	\begin{enumerate}
	\item 
	\begin{itemize} 
		\item $ R$ is reflexive since for any $a \in A$, $9 | a^2 - a^2$, so $a^2 \equiv b^2 \; [9]$, i.e. $a  R a$.
		\item $ R$ is symmetric since for any $a,b \in A$, if $a  R b$, then $9 | a^2 - b^2$, so $9 | - (a^2 - b^2) = b^2 - a^2$, so $b  R a$.
		\item $ R$ is transitive since for any $a,b, c \in A$, if $a  R b$ and $b  R c$, then $9 | a^2 - b^2$ and $9 | b^2 - c^2$, so 
		\[
		9 | (a^ 2 - b^2) + (b^2 - c^2) = a^2 - c^2,
		\]
		so $a  R c$.
	\end{itemize}
	\item Now, $[0]$ contains all $a \in A$ such that $9 | a^2$, which is only true for $a = 0, 3, 6$. Then $[1]$ contains all $a \in A$ such that $9 | a^2 - 1$, which is only true for $a = 1, 8$. Then $[2]$ contains all $a \in A$ such that $9 | a^2 - 4$, which is only true for $a = 2, 7$. Then $[4]$ contains all $a \in A$ such that $9 | a^2 - 16$, which is only true for $a = 4, 5$. We see that we have found all elements, and therefore there are four equivalence classes:
		\begin{align*}
		[0] = [3] &= \{ 0, 3, 6 \} \\
		[1] = [8] &= \{ 1, 8 \}
		[2] = [7] &= \{ 2, 7 \}
		[4] = [5] &= \{ 4, 5 \}
		\end{align*}
	\end{enumerate}
\end{solution}


\begin{question}
    Are the given relations reflexive? antisymmetric? transitive? Either \textit{prove} generally or \textit{disprove} via 
    counterexample.
    	\begin{enumerate}
	\item For $x, y \in \Z$,  $x R y \iff |x - y| \leq 2$. 
	\item  $x R y$ means that $x$ and $y$ have a common prime factor (a prime number that divides both $x$ and $y$), 
	where $x, y \in \Z$.
	\item For $x, y \in 2^{\Z}$. $x R y \iff x \cap y \neq \emptyset$.
	\end{enumerate}
\end{question}
% Student: put your answer between the next two lines.
\begin{solution}
\begin{enumerate}
	\item
\begin{itemize}
		\item $ R$ is reflexive since for any $x \in \Z$, $|x-x|  = 0 \leq 2$.
		\item $ R$ is not antisymmetric. Take $x = 0$ and $y = 1$. Then clearly $x  R y$ and $y  R x$, but $x \neq y$.
		\item $ R$ is not transitive. Indeed, take $x = 0$, $y = 2$, and $z = 4$. Then clearly $x  R y$ and $y  R z$, but $x$ and $z$ are not related.
	\end{itemize}
	
	\item 
	\begin{itemize}
		\item $ R$ is not reflexive, since for instance $1$ does not have a prime factor.
		\item $ R$ is not antisymmetric. Indeed, take $x = 2$ and $y = 4$. Then $x$ and $y$ have a common prime factor 2, so $x  R y$ and $y  R x$, but $x \neq y$.
		\item $ R$ is not transitive. Indeed, take $x = 2$, $y = 6$, and $z = 3$. Then $x  R y$ since $x$ and $y$ both have the prime factor 2, and $y  R z$ since $y$ and $z$ both have the prime factor 3, but $x$ and $z$ do not have a prime factor.
	\end{itemize}
	
	\item 
	\begin{itemize}
		\item $ R$ is not reflexive, since $\emptyset \cap \emptyset = \emptyset$.
		\item $ R$ is not antisymmetric. Indeed, take $x = \{ 0 \}$ and $y = \{0,1\}$. Then $x \cap y = \{0 \} \neq \emptyset$, but $x \neq y$.
		\item $ R$ is not transitive. Indeed, take $x = \{ 0 \}$, $y = \{0,1\}$, and $z = \{1\}$. Then clearly $x  R y$ and $y  R z$, but $x$ and $z$ are not related.
	\end{itemize}
     \end{enumerate}
\end{solution}


\begin{question}
    (Scheinerman, Exercise 15.11:)
    Suppose that $R$ is an equivalence relation on  set $A$, and suppose that $a, b \in A$.
    Prove: $a \in [b]$ if and only if $b \in [a]$.
\end{question}
% Student: put your answer between the next two lines.
\begin{solution}
    \begin{itemize}
    \item[($\Rightarrow$)] Suppose that $a \in [b]$.  This means that $a R b$.  By symmetry (since $R$ is an equivalence relation), $b R a$.  Therefore, $b \in [a]$.
    \item[($\Leftarrow$)] Suppose that $b \in [a]$.  This means that $b R a$.  By symmetry (since $R$ is an equivalence relation), $a R b$.  Therefore, $a \in [b]$.
    \end{itemize}
\end{solution}


\begin{question}
    Count the following objects.
    	\begin{enumerate}
    	\item The number of anagrams (including nonsensical words) of the word MISSISSIPPI.
   	 \item The number of partitions in two of $\{1, 2, \dots, 100 \}$. Remember, both parts should be non-empty.
   	\end{enumerate}
\end{question}
% Student: put your answer between the next two lines.
\begin{solution}
    	\begin{enumerate}
    	\item There are $11!$ orders if all letters are distinct. However, we could shuffle the I ($4!$ ways), the S ($4!$ ways), or the P ($2!$ ways), and still obtain the same words. Therefore, we have
	\[
	\frac{11!}{2! (4!)^2}
	\]
	different words in total.
   	 \item One of the parts must contain the element 1. Now, for each of the other 99 elements, we can decide whether we want to put them in the same part as 1, or in the other one. This gives 2 choices for each, for a total of $2^{99}$ ways. However, one option is not valid: when we put all other elements with 1 (since then, the other part would be empty). This leaves us with $2^{99} - 1$ partitions in two parts.
	
	An alternative way to see this is the following: choosing a partition in two parts is like choosing a subset ($2^{100}$ of them). However, $\emptyset$ and the whole sets are not allowed, which leaves us with $2^{100} - 2$ subsets. Then, if we choose a subset $A$ or its complement, we get the same partition, so each partition is double counted, which gives a total of
	\[
	\frac12 \left ( 2^{100} - 2 \right ) = 2^{99} - 1
	\]
	different partitions.
   	\end{enumerate}
\end{solution}




\begin{question}
    \begin{enumerate}
        \item Twenty people are to be divided into two teams with ten players on each team.  In how many ways can this be 	
        done?
        \item Thirty five discrete math students are to be divided into seven discussion groups, each consisting of five students.  
        In how many ways can this be done?
    \end{enumerate}
\end{question}
% Student: put your answer between the next two lines.
\begin{solution}
    \begin{enumerate}
        \item We want to choose 10 people out of 20 people.  There are $\binom{20}{10}$ ways to do this.  (Then, the remaining 10 is in the other team.) 
        Since the two teams are not labelled, then we divide by $2!$.  Thus, the solution is:
        \[ \frac{1}{2} \binom{20}{10} = \frac{20!}{10!10!2!}. \]
        \item There are $\binom{35}{7}$ ways to choose 7 students out of 35 students.  For each way, there are $\binom{28}{7}$ ways to choose 7 students out of the remaining 28.  Then, there are $\binom{21}{7}$ ways to choose 7 students out of the remaining 21.  And finally, there are $\binom{14}{7}$ ways to divide the remaining 14 students into 2 groups of 7 students.  

We use the multiplication principle to find the total number of ways this can be done.  Since the teams are not labelled, then we further divide the result by $7!$.

Therefore,
\[ 
    \frac{1}{7!} \,\binom{35}{7} \cdot \binom{28}{7} \cdot \binom{21}{7 } \cdot \binom{14}{7} = 
    \frac{1}{7!} \, \frac{35!}{7!21!} \cdot \frac{21!}{7!14!} \cdot \frac{14!}{7!7!}
    = \frac{35!}{(7!)^5}
\]
    \end{enumerate}
\end{solution}


\begin{question}
    (Scheinerman, Exercise 17.16:)
    Prove \textbf{combinatorially} that
    \[ k \binom{n}{k}  = n \binom{n-1}{k-1}. \]
\end{question}
% Student: put your answer between the next two lines.
\begin{solution}
\textbf{Question}: We would like to form a committee consisting of $k$ people chosen from $n$ possible candidates, together with one committee chair chosen from the $k$ committee members.  How many possible committee-and-chair combinations are there?

\textbf{Answer 1}: There are $\binom{n}{k}$ ways to choose a committee of size $k$ out of $n$ possible candidates.  For each choice of $k$ committee-members, there are $k$ possible ways to choose one chair.  Therefore, by the multiplication principle, there are $k \binom{n}{k}$ different committee-and-chair combinations.

\textbf{Answer 2}: First, choose one person to be the committee chair from the $n$ possible candidates.  There are $n$ ways to do this.  Then, there are $k-1$ more committee members that should be chosen from the remaining $n-1$ candidates; there are $\binom{n-1}{k-1}$ ways this could be done.  Therefore, by the multiplication principle, there are $n \binom{n-1}{k-1}$ different committee-and-chair combinations.

Thus, the two answers must be equal:
\[ k \binom{n}{k} = n \binom{n-1}{k-1}. \]
\end{solution}


\begin{question}
    (Scheinerman, Exercise 17.26:)
    Prove \textbf{combinatorially} that
    \[ \binom{n}{0}\binom{n}{n} + \binom{n}{1} \binom{n}{n-1} + \binom{n}{2}\binom{n}{n-2} + \ldots + \binom{n}{n-1}\binom{n}{1} + \binom{n}{n}\binom{n}{0} = \binom{2n}{n}. \]
\end{question}
% Student: put your answer between the next two lines.
\begin{solution}
\textbf{Question}: There are $n$ red balls numbered $1, 2, \ldots, n$, and $n$ green balls numbered $1, 2, \ldots, n$.  We would like to count the number of ways we can choose $n$ balls out of the above $2n$ balls.

\textbf{Answer 1}: There are $\binom{2n}{n}$ ways to choose $n$ items out of a collection of $2n$ items.

\textbf{Answer 2:} Out of the $n$ balls that we choose, some might be red balls and the remaining are green balls.

The number of ways we can choose $n$ balls such that $0$ balls are red and $n$ balls are green is: the number of ways we can choose 0 out of $n$ red balls times the number of ways we can choose $n$ out of $n$ green balls:
\[ \binom{n}{0} \binom{n}{n}.\]

The number of ways we can choose $n$ balls such that $1$ ball is red and $n-1$ balls are green is: the number of ways we can chose 1 out of $n$ red balls times the number of ways we can choose $n-1$ out of $n$ green balls:
\[ \binom{n}{1} \binom{n}{n-1}. \]

That is, for each $k \in \{0, 1, \ldots, n\}$, the number of ways we can choose $n$ balls such that $k$ of them are red and $n-k$ are green is:
\[ \binom{n}{k}\binom{n}{n-k}. \]
Therefore, the total number of ways we can choose $n$ balls out of $2n$ balls is:
\[ \sum_{k=0}^n \binom{n}{k}\binom{n}{n-k} = \binom{n}{0}\binom{n}{n} + \binom{n}{1} \binom{n}{n-1} + \ldots + \binom{n}{n}\binom{n}{0} \]


Since the above two are correct answers to the same question, then they must be equal:
\[ \binom{n}{0}\binom{n}{n} + \binom{n}{1} \binom{n}{n-1} + \binom{n}{2}\binom{n}{n-2} + \ldots + \binom{n}{n-1}\binom{n}{1} + \binom{n}{n}\binom{n}{0} = \binom{2n}{n}. \]
\end{solution}


\end{document}
