\documentclass{article}
% The LaTeX macro language is complicated, so we have inserted
% lots of documenting comments into the file.  Comments start
% with `%' and continue to the end of the line.  In Overleaf's
% window, they are colored green
%
% Comments prefixed with `Student:' are relevant to students.
% Skip anything else you don't understand, or ask me.
%
% set font encoding for PDFLaTeX or XeLaTeX
\usepackage{ifxetex}
\ifxetex
  \usepackage{fontspec}
\else
  \usepackage[T1]{fontenc}
  \usepackage[utf8]{inputenc}
  \usepackage{lmodern}
\fi

% Student: These lines describe some document metadata.
\title{Problem Set 4}
\author{%
% Student: change the next line to your name!
    Name
\\  MATH-UA 120 Discrete Mathematics
}
\date{due November 11, 2022 at 11:00pm}


\usepackage[headings=runin-fixed-nr]{exsheets}
% These make enumerates within questions start at the second ("(a)") level, rather than the first ("1.") level.
\makeatletter
    \newcommand{\stepenumdepth}{\advance\@enumdepth\@ne}
\makeatother
\SetupExSheets{
    question/pre-body-hook=\stepenumdepth,
    solution/pre-body-hook=\stepenumdepth,
}
\DeclareInstance{exsheets-heading}{runin-nn-np}{default}{
    runin = true,
    title-post-code = .\space,
    join = {
        main[r,vc]title[l,vc](0pt,0pt);
    }
}
\newif\ifshowsolutions
% Student: replace `false' with `true' to typeset your solutions.
% Otherwise they are ignored!
\showsolutionsfalse
\ifshowsolutions
    \SetupExSheets{
        question/pre-hook=\itshape,
        solution/headings=runin-nn-np,
        solution/print=true,
        solution/name=Answer
    }%
    \makeatletter%
    \pretocmd{\@title}{Answers to }%
    \makeatother%
\else
    \SetupExSheets{solution/print=false}
\fi

% Bug workaround: http://tex.stackexchange.com/a/146536/1402
%\newenvironment{exercise}{}{}
\RenewQuSolPair{question}{solution}
%\let\answer\solution
%\let\endanswer\endsolution
\usepackage{manfnt}
\newcommand{\danger}{\marginpar[\hfill\dbend]{\dbend\hfill}}

% We are creating a command for some common commands.
\newcommand{\Z}{\mathbb{Z}}

% This package is for specifying graphics.  It's amazing.
% Manual at http://texdoc.net/texmf-dist/doc/generic/pgf/pgfmanual.pdf
\usepackage{tikz}

\usepackage{amsmath, amsthm}
\usepackage{amsfonts}
\usepackage{siunitx}
\DeclareSIUnit\pound{lb}
\usepackage{hyperref}
\newtheorem*{theorem}{Theorem}
\theoremstyle{definition}
\newtheorem*{definition}{Definition}
% This is the beginning of the part of the file that describes
% the text of the document.
% That's why it says `\begin{document}' below. :-)
\begin{document}
\maketitle



These are to be written up in \LaTeX{} and turned in to Gradescope.\\



\ifshowsolutions
    \SetupExSheets{solution/print=true}
\else
    \danger
 \underline{ \LaTeX{}  Instructions:}  You can view the source (\texttt{.tex}) file to get some more examples of \LaTeX{} code.  I have commented the source file in places where new \LaTeX{} constructions are used.
  
  Remember to change \verb|\showsolutionsfalse| to \verb|\showsolutionstrue|
    in the document's preamble 
    (between \verb|\documentclass{article}| and \verb|\begin{document}|)
\fi

\section*{Assigned Problems}

\begin{question}
    Prove the following statement by contrapositive: \\
    For all $n\in \mathbb{N}$, if $2^n<n!$, then $n>3$.
\end{question}
% Student: put your answer between the next two lines.
\begin{solution}
\end{solution}


\begin{question}
    Prove the following statement by contradiction:\\
    There exists no integers $a$ and $b$ for which $21a + 30b = 1$.
\end{question}
% Student: put your answer between the next two lines.
\begin{solution}
\end{solution}




\begin{question}
    (Scheinerman, Exercise 20.15:)
    Prove the following statement by contradiction:\\ Let $A$ and $B$ be sets.  Prove that if $|A \cup B| = |A| + |B|$, then $A\cap B = \emptyset$.
\end{question}
% Student: put your answer between the next two lines.
\begin{solution}
\end{solution}


\begin{question}
    (Scheinerman, Exercise 21.7:) 
    The inequality $F_n > 1.6^n$ is true for all $n$ that are large enough.  Do some calculations to find out for what values of $n$ this inequality holds.  Then, prove your assertion using contradiction and smallest counterexample.

    (Note: you want to prove a statement of the form: ``For all integers $n \geq N$, $F_n > 1.6^n$.''  Your task is to first figure out what number $N$ should be.)\\~\\
    \textit{Hint: $N=29$.}
\end{question}
% Student: put your answer between the next two lines.
\begin{solution}
\end{solution}


\begin{question}
    (Scheinerman, Exercise 22.4:) 
    Prove the following by induction.  For a positive integer $n$, the $n$th derivative of $x^n$ is $n!$; that is $ \displaystyle \frac{d^n}{dx^n} x^n = n!$.
\end{question}
% Student: put your answer between the next two lines.
\begin{solution}
% Here is an example on how to label your bullets to follow the induction template.
%\begin{description}
%\item[Base Case: ] 
%\item[Inductive Hypothesis: ] 
%\item[Inductive Step: ] 
%\end{description}
\end{solution}


\begin{question}
    (Scheinerman, Exercise 22.12:) 
    The Tower of Hanoi is a puzzle consisting of a board with three dowels and a collection of $n$ disks of $n$ different sizes (radii).  The disks have holes drilled through their centers so that they can fit on the dowels on the board.  Initially all disks are on the first dowel and tare arranged in size order (from the largest on the bottom to the smallest on the top).

    The object is to move all the disks to another dowel in as few moves as possible.  Each move consists of taking the top disk off one of the stacks and placing it on another stack, with the added condition that you may not place a larger disk atop a smaller one.  The figure in ``Lecture 9 Worksheet'' shows how to solve the Tower of Hanoi in seven moves when $n = 3$.
    
    \textbf{Prove:}  For every positive integer $n$, the Tower of Hanoi puzzle with $n$ disks can be solved in $2^n-1$ moves.
\end{question}
% Student: put your answer between the next two lines.
\begin{solution}
\end{solution}



\begin{question}
    Use strong induction to show that postage of 24-cents or more can be achieved by using only 5-cent and 7-cent stamps.
\end{question}
% Student: put your answer between the next two lines.
\begin{solution}
\end{solution}

\begin{question}
    (Scheinerman, Exercise 22.24:)
    Use strong induction to show that every natural number can be expressed as the sum of distinct powers of $2$.  (For example, $21 = 2^0 + 2^2 + 2^ 4$.)
\end{question}
% Student: put your answer between the next two lines.
\begin{solution}
\end{solution}


\begin{question}
    Use strong induction to show if $n\in \mathbb{N}$, then $12\mid (n^4-n^2)$. \textit{Hint: This involves a lot of algebra. Try to simplify and then use the binomial theorem.}
\end{question}
% Student: put your answer between the next two lines.
\begin{solution}
\end{solution}



\begin{question}
    Suppose that we have two piles of cards each containing $n$ cards. Two players play a game as follows. Each player, in turn, chooses one pile and them removes any number of cards. but at least one, from the chosen pile. The player who removes the last card wins the game. Show that the second player can always win the game.
\end{question}
% Student: put your answer between the next two lines.
\begin{solution}
\end{solution}


\end{document}
