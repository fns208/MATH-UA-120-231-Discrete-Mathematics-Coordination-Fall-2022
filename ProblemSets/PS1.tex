\documentclass{article}
% The LaTeX macro language is complicated, so we have inserted
% lots of documenting comments into the file.  Comments start
% with `%' and continue to the end of the line.  In Overleaf's
% window, they are colored blue.
%
% Comments prefixed with `Student:' are relevant to students.
% Skip anything else you don't understand, or ask me.
%
% set font encoding for PDFLaTeX or XeLaTeX
\usepackage{ifxetex}
\ifxetex
  \usepackage{fontspec}
\else
  \usepackage[T1]{fontenc}
  \usepackage[utf8]{inputenc}
  \usepackage{lmodern}
\fi

% Student: These lines describe some document metadata.
\title{Problem Set 1}
\author{%
% Student: change the next line to your name!
    Name
\\  MATH-UA 120 Discrete Mathematics
}
\date{due September 23, 11pm on Gradescope}


\usepackage[headings=runin-fixed-nr]{exsheets}
% These make enumerates within questions start at the second ("(a)") level, rather than the first ("1.") level.
\makeatletter
    \newcommand{\stepenumdepth}{\advance\@enumdepth\@ne}
\makeatother
\SetupExSheets{
    question/pre-body-hook=\stepenumdepth,
    solution/pre-body-hook=\stepenumdepth,
}
\DeclareInstance{exsheets-heading}{runin-nn-np}{default}{
    runin = true,
    title-post-code = .\space,
    join = {
        main[r,vc]title[l,vc](0pt,0pt);
    }
}
\newif\ifshowsolutions
% Student: replace `false' with `true' to typeset your solutions.
% Otherwise they are ignored!
\showsolutionsfalse
\ifshowsolutions
    \SetupExSheets{
        question/pre-hook=\itshape,
        solution/headings=runin-nn-np,
        solution/print=true,
        solution/name=Answer
    }%
    \makeatletter%
    \pretocmd{\@title}{Answers to }%
    \makeatother%
\else
    \SetupExSheets{solution/print=false}
\fi

% Bug workaround: http://tex.stackexchange.com/a/146536/1402
%\newenvironment{exercise}{}{}
\RenewQuSolPair{question}{solution}
%\let\answer\solution
%\let\endanswer\endsolution
\usepackage{manfnt}
\newcommand{\danger}{\marginpar[\hfill\dbend]{\dbend\hfill}}

\usepackage{amsmath, amsthm}
\usepackage{amsfonts}
\usepackage{siunitx}
\DeclareSIUnit\pound{lb}
\usepackage{hyperref}
\newtheorem*{theorem}{Theorem}
\theoremstyle{definition}
\newtheorem*{definition}{Definition}
% We are creating a command "\xor".
\newcommand{\xor}{\underline{\lor}}

% This is the beginning of the part of the file that describes
% the text of the document.
% That's why it says `\begin{document}' below. :-)
\begin{document}
\maketitle



These are to be written up and turned in to Gradescope.\\



\ifshowsolutions
    \SetupExSheets{solution/print=true}
\else
    \danger
 \underline{ \LaTeX  Instructions:}  You can view the source (\texttt{.tex}) file to get some more examples of \LaTeX{} code.  I have commented the source file in places where new \LaTeX{} constructions are used.
  
  Remember to change \verb|\showsolutionsfalse| to \verb|\showsolutionstrue|
    in the document's preamble 
    (between \verb|\documentclass{article}| and \verb|\begin{document}|)
\fi

\section*{Assigned Problems}

\begin{question}
Determine if the following statement is true or false and support your reasoning. \emph{Your reasoning must be specific to this statement.}
\begin{quote}
``Every positive integer is either prime or composite.''
\end{quote}
\end{question}
% Student: put your answer between the next two lines.
\begin{solution}
\end{solution}

\begin{question}
    % Notice the use of the enumerate environment
    % to make a numbered list.  Each item is marked
    % by the \item command.
    %
    % Also \emph = emphasize, usually in italics.
    \begin{enumerate}
        \item 
        Suppose that the concept of points and lines
        in the plane is already defined.  Write a careful
        definition for two lines to be \emph{parallel}. 
        Your definition should begin as
        follows:
        % What do you think center does?
        \begin{center}
        Suppose $l$ and $m$ are lines in the plane. We say that $l$ and $m$ are \emph{parallel} provided...
        \end{center}

        {\emph{Note:} Since you are crafting this definition, you have a bit of flexibility.  Consider excluding the possibility that the line $l$ might be the same as the line $m$. Be sure that your definition does what you intend.}

        \item Using your definition of \emph{parallel}, explain why the following lines are \textbf{not} parallel: $y=2x+1$ and $y=x+2$.
    \end{enumerate}
\end{question}
% Student: put your answer between the next two lines.
\begin{solution}
\end{solution}


\begin{question}
   (Scheinerman, Exercise 4.2:)
    Below you will find pairs of statements $A$ and $B$. For each pair, please indicate which
of the following three sentences are true and which are false:
    \begin{itemize}
        \item If $A$, then $B$.
        \item If $B$, then $A$.
        \item $A$ if and only $B$.
    \end{itemize}
    \begin{enumerate}
        \item $A$: Polygon $PQRS$ is a rectangle. 
              $B$: Polygon $PQRS$ is a parallelogram.
        \item $A$: Joe is a grandfather.
              $B$: Joe is male.

         For the remaining items, $x$ and $y$ refer to \emph{real} numbers.

         \item $A$: $x>0$
               $B$: $x^2>0$
         \item $A$: $x<0$
               $B$: $x^3<0$
               
    \end{enumerate}
\end{question}
% Student: put your answer between the next two lines.
\begin{solution}
\end{solution}


\begin{question}
    (Scheinerman, Exercise 5.11:)
    Suppose $a$, $b$, $d$, $x$, and $y$ are integers.
    Prove that if $d \mid a$ and $d\mid b$,
    then $d \mid (ax+by)$.
\end{question}
% Student: put your answer between the next two lines.
\begin{solution}
\end{solution}



\begin{question}
Consider the following definition of the ``$\triangleleft$'' symbol.
	\begin{definition}
	Let $x$ and $y$ be integers. Write $x\triangleleft y$ if $3x+5y=7k$ for some integer $k$.
	\end{definition}
\begin{enumerate}
\item Show that $1\triangleleft 5$, $3\triangleleft 1$, and $0\triangleleft 7$.
\item Prove that if $a\triangleleft b$ and $c\triangleleft d$, then $(a+c) \triangleleft (b+d)$.
\end{enumerate}
\end{question}
% Student: put your answer between the next two lines.
\begin{solution}
\end{solution}

\begin{question}
    Show that an integer $n$ is odd if and only if $2n+2$ is divisible by 4.
\end{question}
% Student: put your answer between the next two lines.
\begin{solution}
\end{solution}

\begin{question}
    \begin{enumerate}
        \item Disprove: For all real numbers $x$, $x^2\geq x$.
        \item Disprove: For every positive nonprime integers $n$, if some prime $p$ divides $n$, then some other prime $q$ ($q\neq p$) also divides $n$.
    \end{enumerate}
\end{question}
% Student: put your answer between the next two lines.
\begin{solution}
\end{solution}

\begin{question}
   (Scheinerman, Exercise 7.12:)
    Another method to prove that certain Boolean formulas
    are tautologies is to use the properties in Theorem~7.2
    together with the fact that $x \rightarrow y$ is equivalent
    to $(\neg x) \vee y$ (Proposition~7.3)
    For example, Exercise 7.11, part (b) asks you to establish that the formula $(x \wedge (x \rightarrow y)) \rightarrow y$ is 
    a tautology.  Here is a derivation of that fact:
    % This is an "align" environment for aligning
    % mathematical equations.  Ampersands (&) denote 
    % separation between columns.  The first one means
    % there will be alignment around the equal sign.
    % The double-ampersands put a second column, left-
    % justified.  Double-backslashes (\\) separate lines.
    \begin{align*}
        (x \wedge (x \rightarrow y)) \rightarrow y
        &= [x \wedge (\neg x \vee y)] \rightarrow y
        && \text{translate $\rightarrow$} 
        \\
        &= [(x \wedge \neg x) \vee (x \wedge y)] \rightarrow y
        && \text{distributive}
        \\
        &= [\mathrm{FALSE} \vee (x\wedge y)] \rightarrow y
        && \text{inverse elements}
        \\
        &= (x\wedge y) \rightarrow y
        && \text{identity element}
        \\
        &= \neg(x\wedge y) \vee y
        && \text{translate $\rightarrow$}
        \\
        &= (\neg x \vee \neg y) \vee y
        && \text{De~Morgan's laws}
        \\
        &= \neg x \vee (\neg y \vee y)
        && \text{associativity}
        \\
        &= \neg x \vee \mathrm{TRUE}
        && \text{inverse elements}
        \\
        &= \mathrm{TRUE}
        && \text{identity element}
        \\
    \end{align*}
    Use this technique [not truth tables] to prove that these formulas
    are tautologies:
    \begin{enumerate}
        \item $(x \rightarrow \mathrm{FALSE}) \rightarrow \neg x$
        \item $((x \rightarrow y) \wedge (x \rightarrow \neg y)) \rightarrow \neg x$
    \end{enumerate}
\end{question}
% Student: put your answer between the next two lines.
\begin{solution}
\end{solution}




\begin{question}
   (Scheinerman, Exercise 7.16:)
Here is another Boolean operation called \textit{exclusive or}; it is denoted by the symbol $\xor$. It is defined in the following table.
	% This is an "array" environment for aligning
	% individual objects. It is similar to "align"
	% Right after we \begin{array}, we indicate
	% how we want to align each individual column:
	% "c" for centered, "r" for right-justified,
	% "l" ("ell") for left-justified.
	% The total number of c's, r's, and l's should match
	% the number of columns.
	% We can use this to form a table. To create the borders,
	% we use "|" (the notation for "divides") to create a
	% vertical line between each column.
	% After each line, double-backslashes (\\) separate lines.
	% If we want a horizontal line after each line, 
	% we use "\hline".
	%For math environment, we can use "$$" or "\[ \]".
	\[\begin{array}{| c | c || c |}
	\hline
	x & y & x \xor y \\
	\hline
	T & T & F \\ 
	T & F & T \\
	F & T & T \\
	F & F & F \\
	\hline
	\end{array}\]
    \begin{enumerate}
        \item Prove that $x \xor y$ is logically equivalent to $(x \wedge \neg y) \vee ((\neg x) \wedge y)$.
        \item Prove that $x \xor y$ is logically equivalent to $(x \vee y) \wedge (\neg (x \wedge y))$.
        \item Explain why the operation $\xor$ is called \textit{exclusive or}.
    \end{enumerate}
\end{question}
% Student: put your answer between the next two lines.
\begin{solution}
\end{solution}

\begin{question}
	Let the following statements be given.
		\begin{align*}
		p &= \text{``Jeremiah is hungry.''}\\
		q &= \text{``The refrigerator is empty.''}\\
		r &= \text{``Jeremiah is mad.''}
		\end{align*}
	\begin{enumerate}
		\item Rewrite the following statement as a Boolean expression.\\
			\begin{quote}
			If Jeremiah is hungry and the refrigerator is empty, then Jeremiah is mad.
			\end{quote}
		\item Construct a truth table for the statement in part (a).
		\item Suppose that the statement given in part (a) is true, and suppose also that Jeremiah is not mad and the refrigerator is empty. Is Jeremiah hungry? Justify your answer using the truth table.
	\end{enumerate}
\end{question}
% Student: put your answer between the next two lines.
\begin{solution}
\end{solution}

\end{document}
