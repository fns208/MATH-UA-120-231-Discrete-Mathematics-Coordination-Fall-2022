\documentclass{article}
% The LaTeX macro language is complicated, so we have inserted
% lots of documenting comments into the file.  Comments start
% with `%' and continue to the end of the line.  In Overleaf's
% window, they are colored green
%
% Comments prefixed with `Student:' are relevant to students.
% Skip anything else you don't understand, or ask me.
%
% set font encoding for PDFLaTeX or XeLaTeX
\usepackage{ifxetex}
\ifxetex
  \usepackage{fontspec}
\else
  \usepackage[T1]{fontenc}
  \usepackage[utf8]{inputenc}
  \usepackage{lmodern}
\fi

% Student: These lines describe some document metadata.
\title{Problem Set 3}
\author{%
% Student: change the next line to your name!
    Name
\\  MATH-UA 120 Discrete Mathematics
}
\date{due October 28, 2022 at 11:00pm}


\usepackage[headings=runin-fixed-nr]{exsheets}
% These make enumerates within questions start at the second ("(a)") level, rather than the first ("1.") level.
\makeatletter
    \newcommand{\stepenumdepth}{\advance\@enumdepth\@ne}
\makeatother
\SetupExSheets{
    question/pre-body-hook=\stepenumdepth,
    solution/pre-body-hook=\stepenumdepth,
}
\DeclareInstance{exsheets-heading}{runin-nn-np}{default}{
    runin = true,
    title-post-code = .\space,
    join = {
        main[r,vc]title[l,vc](0pt,0pt);
    }
}
\newif\ifshowsolutions
% Student: replace `false' with `true' to typeset your solutions.
% Otherwise they are ignored!
\showsolutionsfalse
\ifshowsolutions
    \SetupExSheets{
        question/pre-hook=\itshape,
        solution/headings=runin-nn-np,
        solution/print=true,
        solution/name=Answer
    }%
    \makeatletter%
    \pretocmd{\@title}{Answers to }%
    \makeatother%
\else
    \SetupExSheets{solution/print=false}
\fi

% Bug workaround: http://tex.stackexchange.com/a/146536/1402
%\newenvironment{exercise}{}{}
\RenewQuSolPair{question}{solution}
%\let\answer\solution
%\let\endanswer\endsolution
\usepackage{manfnt}
\newcommand{\danger}{\marginpar[\hfill\dbend]{\dbend\hfill}}

% We are creating a command for some common commands.
\newcommand{\Z}{\mathbb{Z}}

% This package is for specifying graphics.  It's amazing.
% Manual at http://texdoc.net/texmf-dist/doc/generic/pgf/pgfmanual.pdf
\usepackage{tikz}

\usepackage{amsmath, amsthm}
\usepackage{amsfonts}
\usepackage{siunitx}
\DeclareSIUnit\pound{lb}
\usepackage{hyperref}
\newtheorem*{theorem}{Theorem}
\theoremstyle{definition}
\newtheorem*{definition}{Definition}
% This is the beginning of the part of the file that describes
% the text of the document.
% That's why it says `\begin{document}' below. :-)
\begin{document}
\maketitle



These are to be written up in \LaTeX{} and turned in to Gradescope.\\



\ifshowsolutions
    \SetupExSheets{solution/print=true}
\else
    \danger
 \underline{ \LaTeX{}  Instructions:}  You can view the source (\texttt{.tex}) file to get some more examples of \LaTeX{} code.  I have commented the source file in places where new \LaTeX{} constructions are used.
  
  Remember to change \verb|\showsolutionsfalse| to \verb|\showsolutionstrue|
    in the document's preamble 
    (between \verb|\documentclass{article}| and \verb|\begin{document}|)
\fi

\section*{Assigned Problems}

\begin{question}
    (Scheinerman, Exercise 14.15:)
    Prove: A relation $R$ on a set $A$ is antisymmetric if 
    \[ R \cap R^{-1} \subseteq \{ (a, a) : a \in A \}. \]
\end{question}
% Student: put your answer between the next two lines.
\begin{solution}
\end{solution}


\begin{question}
    (Scheinerman, Exercise 14.16:)
    \begin{enumerate}
        \item Give an example of a relation on a set that is both symmetric and transitive, but not reflexive.
        \item Explain what is wrong with the following `proof':
        
        \fbox{\parbox{\linewidth}{
    	\textbf{Statement:} If $R$ is symmetric and transitive, then $R$ is reflexive.

	\textbf{``Proof'':} Suppose $R$ is symmetric and transitive.  Symmetric means that $x\mathrel{R}y$ implies $y \mathrel{R}x$.  We apply transitivity to $x \mathrel{R}y$ and $y \mathrel{R}x$ to give $x \mathrel{R}x$.  Therefore, $R$ is reflexive.
        }}
    \end{enumerate}
\end{question}
% Student: put your answer between the next two lines.
\begin{solution}
\end{solution}


\begin{question}
    (Scheinerman, Exercise 15.4:)
    \begin{enumerate}
        \item Prove that if $x$ and $y$ are both odd integers, then $x \equiv y \pmod{2}$.
        \item Prove that if $x$ and $y$ are both even integers, then $x \equiv y \pmod{2}$.
    \end{enumerate}
\end{question}
% Student: put your answer between the next two lines.
\begin{solution}
\end{solution}


\begin{question}
    Consider the set $A = \{0, 1, 2, \dots, 8 \}$. Define a relation $R$ on $A$ by
	\[
	a\mathrel{R}b \iff a^2 \equiv b^2 \pmod{9}.
	\]
	\begin{enumerate}
	\item Show that $R$ is an equivalence relation, 
	\item then determine all its (distinct) equivalence classes.
	\end{enumerate}
\end{question}
% Student: put your answer between the next two lines.
\begin{solution}
\end{solution}


\begin{question}
    Are the given relations reflexive? antisymmetric? transitive? Either \textit{prove} generally or \textit{disprove} via 
    counterexample.
    	\begin{enumerate}
	\item For $x, y \in \Z$,  $x\mathrel{R}y \iff |x - y| \leq 2$. 
	\item  $x\mathrel{R}y$ means that $x$ and $y$ have a common prime factor (a prime number that divides both $x$ and $y$), 
	where $x, y \in \Z$.
	\item For $x, y \in 2^{\Z}$. $x\mathrel{R}y \iff x \cap y \neq \emptyset$.
	\end{enumerate}
\end{question}
% Student: put your answer between the next two lines.
\begin{solution}
\end{solution}


\begin{question}
    (Scheinerman, Exercise 15.11:)
    Suppose that $R$ is an equivalence relation on  set $A$, and suppose that $a, b \in A$.
    Prove: $a \in [b]$ if and only if $b \in [a]$.
\end{question}
% Student: put your answer between the next two lines.
\begin{solution}
\end{solution}


\begin{question}
    Count the following objects.
    	\begin{enumerate}
    	\item The number of anagrams (including nonsensical words) of the word MISSISSIPPI.
   	 \item The number of partitions in two of $\{1, 2, \dots, 100 \}$. Remember, both parts should be non-empty.
   	\end{enumerate}
\end{question}
% Student: put your answer between the next two lines.
\begin{solution}
\end{solution}




\begin{question}
    \begin{enumerate}
        \item Twenty people are to be divided into two teams with ten players on each team.  In how many ways can this be 	
        done?
        \item Thirty five discrete math students are to be divided into seven discussion groups, each consisting of five students.  
        In how many ways can this be done?
    \end{enumerate}
\end{question}
% Student: put your answer between the next two lines.
\begin{solution}
\end{solution}


\begin{question}
    (Scheinerman, Exercise 17.16:)
    Prove \textbf{combinatorially} that
    \[ k \binom{n}{k}  = n \binom{n-1}{k-1}. \]
\end{question}
% Student: put your answer between the next two lines.
\begin{solution}
\end{solution}


\begin{question}
    (Scheinerman, Exercise 17.26:)
    Prove \textbf{combinatorially} that
    \[ \binom{n}{0}\binom{n}{n} + \binom{n}{1} \binom{n}{n-1} + \binom{n}{2}\binom{n}{n-2} + \dots + \binom{n}{n-1}\binom{n}{1} + \binom{n}{n}\binom{n}{0} = \binom{2n}{n}. \]
\end{question}
% Student: put your answer between the next two lines.
\begin{solution}
\end{solution}


\end{document}
